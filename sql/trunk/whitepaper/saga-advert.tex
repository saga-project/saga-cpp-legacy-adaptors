\documentclass[a4paper,10pt,twocolumn]{article}
%\documentclass[a4paper,10pt]{article}
\usepackage[pdftex]{graphicx}
\usepackage{float}
\usepackage{times}
\usepackage{color}
%\usepackage{fancyheadings}
%\restylefloat{figure}

%\oddsidemargin=0.1in
\topmargin=-0.8in
\textheight=9.8in
\textwidth=6.25in

\parskip 10pt
\parindent 0in

%\parskip 0pt
%\parindent 0.25in
\newif\ifdraft
%\drafttrue


\ifdraft
\newcommand{\fixme}[1]{ { \bf{ ***FIXME: #1 }} }
\newcommand{\note}[1]{ {\textcolor{red} { ***Jha: #1 }}}
\else
\newcommand{\jhanote}[1]{}
\newcommand{\note}[1]{}
\fi


\begin{document}
\thispagestyle{plain}
\title{The SAGA Advert Service: Concepts, Implementations and Usage Modes}
\author{Hans-Christian~Wilhelm \footnotemark, Ole~Weidner\footnotemark \\ {\em \small{Center for Computation and Technology, Louisiana State University,}} \\ {\em {\small 216 Johnston Hall, Baton Rouge, LA 70803, USA}}
\\ {\footnotesize $^*$ hcwilhelm@cct.lsu.edu, $^\dag$ oweidner@cct.lsu.edu}}

\date{}

\maketitle


\begin{abstract}
In general we need to lower the emphasis on SAGA, at least in
the opening part as it has already been published. We want to use this
abstract to i) Define the Scientific Problem that we will be addressing, 
ii) 
\end{abstract}

\section{Introduction} 

\section{Implementations} 

\subsection{Old}

\subsection{New}

\subsection{Comparison}

\thispagestyle{plain}
\end{document} 
