
\newcommand{\sagadocument}{SAGA-iRODS Adaptor}
\newcommand{\sagaversion}{1.0}
\newcommand{\sagabasename}{saga-irods-spec}
\newcommand{\sagaemail}{yutaka.kawai@kek.jp}

\input{saga_include}

% \sagafinal

\newcommand{\name}{\F{SAGA}\xspace}
\DefineShortVerb{\|}

\begin{document}

 \thispagestyle{empty}

 \sagadocument{}\hfill  Yutaka Kawai, KEK\\
  Version: \sagaversion \hfill {\sagadate}

%  \footnotetext[1]{editor}

  \hrulefill\\[2em]

  \B{\large The SAGA-iRODS Adaptor Specification}\\[4em]

%  \U{Copyright Notice}

%  Copyright \copyright~Yutaka Kawai (2010).  All Rights
%  Reserved.\\

%  \U{Abstract}


   This document describes the specification for the SAGA-iRODS Adaptor (SIA).
   

   
   
  \U{Status of This Document}

  This guide is still work in progress.\\

  \newpage

  \tableofcontents

  \newpage

%-----------------------------------------------------------------
% Intro, structure, disclaimer, ...
%-----------------------------------------------------------------
%                                        
%  \sagasec {Introduction                         }{intro}
%  \sagasec {Quick Start Guide                    }{quick} 
%  \sagasec {Building and Running your Application}{make}
%
%
%  \sagapart{General SAGA Concepts                }{general}
%  \newpage
%
%  \sagasec {Error Handling                       }{errors}
%  \sagasec {Using Data Buffers                   }{buffers}
%  \sagasec {Using Attributes                     }{attributes}
%  \sagasec {Using URLs                           }{urls}
%
%
%  \sagapart{SIA SAGA-API                         }{api}
%  \newpage
%
%  \sagasec {SIA File Management                  }{file}
%  \sagasec {SIA Replica Management               }{replica}
  \sagasec {SIA Name Spaces                      }{namespace}

% \appendix
%
% \section*{Appendix}
%
% \sagasec {Complete Code Examples               }{examples}
%
%  \sagabib {saga_manual}
%

\end{document}


